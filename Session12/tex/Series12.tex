
\documentclass[11pt]{article}
\usepackage{../../Shared_Resources/Latex_Styles/General_Style} 
\usepackage{../../Shared_Resources/Latex_Styles/mcode} 

\usepackage[utf8]{inputenc}
\usepackage{hyperref}
\newcommand\bigO[1]{{\ensuremath{\mathcal{O}(#1)}}}
\newtheorem{theorem}{Theorem}
\usepackage{algorithm}
\usepackage{algpseudocode}
\algnewcommand\algorithmicinput{\textbf{Input:}}
\algnewcommand\algorithmicoutput{\textbf{Output:}}
\algnewcommand\Input{\item[\algorithmicinput]}%
\algnewcommand\Output{\item[\algorithmicoutput]}%

\usepackage{listings}

\lstset{ frame=single}

\begin{document}

\lstset{frameround=fttt,language=Matlab}

\lstMakeShortInline[columns=fixed]|

\makeheader{12 -- December 5, 2023}{Arnoldi Process and application to GMRES}


{\bf{Exercise 1: Randomized GS}} \\

Consider the Randomized Gram-Schmidt algorithm (RGS) from Oleg and Grigori, \url{https://arxiv.org/pdf/2011.05090.pdf}.

\scriptsize
\begin{algorithm}
\caption{Randomized Gram-Schmidt algorithm (RGS)}\label{RGS}
\begin{algorithmic}
\Input $n \times m \; W$ and sketching matrix $\Omega$ of size $k \times n$ with $m \leq k \leq n$
\Output $n \times m$ factor $Q$ and $m \times m$ upper triangular factor $R$.
\For{$j = 1:m$}
    \State Sketch $w_i: p_i = \Omega w_i$
    \State Solve the $k \times (i-1)$ least squares problem:
    \[ R_{1:i-1, i} = \text{arg}\min_{y} \|S_{i-1}y - p_i\|. \]
    \State Compute the projection of $w_i: q_i' = w_i - Q_{i-1}R_{1:i-1, i}$.
    \State Sketch $q_i' : s_i' = \Omega q_i'$.
    \State Compute the sketched norm $r_{i,i} = \|s_i'\|$.
    \State Scale vector $s_i = s_i'/r_{i,i}$.
    \State Scale vector $q_i = q_i'/r_{i,i}$
\EndFor
\end{algorithmic}
\end{algorithm}
\normalsize

Suppose you have an orthonormal set $Q_r = \{q_1, q_2, ..., q_r\}$. Implement this algorithm in such way that it orthogonalizes a vector $w_i$ against $Q_r$.

\bigskip

{\bf{Exercise 2: GMRES}} \\

Recall that the Arnoldi process can be used to find eigenvalues. It can also be used to solve systems of equations $Ax^* = b$. The following diagram might be useful:

\begin{figure}[H]
     \centering
     \begin{subfigure}[b]{0.5\textwidth}
         \centering
         \includegraphics[width=\textwidth]{../figures/methods}
     \end{subfigure}
\end{figure}

This was taken from Trefethen and Bau's book \url{http://www.stat.uchicago.edu/~lekheng/courses/309/books/Trefethen-Bau.pdf}.\\


At each step GMRES approximates the solution $x^*$ by a vector in the Krylov subspace $x_n \in \mathcal{K}_n$ that minimizes the residual $r_n = b - Ax_n$. This week we are going to implement GMRES from the lecture slides in a sequential fashion. Next week we are going to modify them so that they are implemented using MPI. 

\begin{enumerate}
   \item Implement Algorithm 2.
   \item Test this method with the following matrices:
   \begin{itemize}
       \item $A$ from the file |VG_mat.csv| and $b$ from the file |VG_b.csv|
       \item $A$ from the file |CTT_mat.csv| and $b$ from the file |CTT_b.csv|
       \item $A$ and $b = f$ from session 9. Such matrix and vector had to do with kernel regression using the MNIST data set.
   \end{itemize}
   \item Plot the number of GMRES iterations vs $L_2$ error
   \item Let $\lambda_1 \geq \lambda_2 \geq ... \geq \lambda_n$ be the eigenvalues of $A$. Plot $\max_{j}\lambda_j/\lambda_{j+1}$ vs the $L_2$ error.
\end{enumerate}


To recall these algorithms:

\scriptsize
\begin{algorithm}
\caption{\textbf{Algorithm 2} GMRES with MGS}\label{GMRES - MGS}
\begin{algorithmic}
\Input $A, x_0, b, \max_{\text{iter}} = m$
\Output $x^*$ approximation to the solution of $Ax^* = b$
\State $r_0 = b - Ax_0$, $\beta = \|r_0\|_2, q_1 = r_0/\beta$
\For{$j = 1:m-1$}
    \State $w_{j+1} = Aq_j$
    \State MGS to orthogonalize $w_{j+1}$ against $\{q_1, ..., q_j\}$
    \State Obtain $[r_0, AQ_j] = Q_{j+1}[\|r_0\|_2 e_1 \bar{H}_j]$
\EndFor
\State Solve $y_m = \text{arg}\min_{y} \| \beta e_1 - \bar{H}_m y \|2$
\end{algorithmic}
\end{algorithm}
\normalsize

You can find the data here: \url{https://www.dropbox.com/scl/fi/fissuao7legmz5cv588sv/Data.zip?rlkey=ccjry33l8bmfaxqwp0odwsby3&dl=0}


\end{document}
