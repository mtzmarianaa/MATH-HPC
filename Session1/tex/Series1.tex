
\documentclass[11pt]{article}
\usepackage{../../Shared_Resources/Latex_Styles/General_Style} 
\usepackage{../../Shared_Resources/Latex_Styles/mcode} 

\usepackage[utf8]{inputenc}
\usepackage{hyperref}
\newcommand\bigO[1]{{\ensuremath{\mathcal{O}(#1)}}}
\newtheorem{theorem}{Theorem}
\usepackage{algorithm}
\usepackage{algpseudocode}
\algnewcommand\algorithmicinput{\textbf{Input:}}
\algnewcommand\algorithmicoutput{\textbf{Output:}}
\algnewcommand\Input{\item[\algorithmicinput]}%
\algnewcommand\Output{\item[\algorithmicoutput]}%

\usepackage{listings}

\lstset{ frame=single}

\begin{document}

\lstset{frameround=fttt,language=Matlab}

\lstMakeShortInline[columns=fixed]|

\makeheader{1 -- September 10, 2024}{Matrices and vectors in Python and MPI}


{\bf{Exercise I Matrices in Python}}\\

Find two efficient ways in Python to assign the following matrix:
\begin{align*}
M = \begin{bmatrix}
    1 & 2 & 3 & 4 \\
    5 & 6 & 7 & 8
\end{bmatrix},
\end{align*}
without entering manually each element (\textit{hint}: create two row vectors and then combine them to form a matrix). 

Suitably use the Python commands to:
\begin{enumerate}
\item extract the element in the first row, third column of $A$;
\item extract the entire second row of $A$;
\item extract the first two columns of $A$;
\item extract the vector containing all the elements of the second row of $A$ except for the third element.
\end{enumerate}

Modify your code accordingly to assign the following matrices as well, calculate the time it takes for Python to assign them in both your methods. Which method is better? Compare your results with one of your classmates.

\begin{align*}
M_{1}  &= \begin{bmatrix} 1 & 2 & 3 & 4& 5 & 6 & 7 & 8 \\ 9 & 10 & 11 & 12 & 13 & 14 & 15 & 16  \end{bmatrix}  \\
M_{2}  &= \begin{bmatrix} 1 & 2 & 3 & 4& 5 & 6 & 7 & 8 \\ 9 & 10 & 11 & 12 & 13 & 14 & 15 & 16 \\ 17 & 18 & 19 & 20 & 21 & 22 & 23 & 24 \\ 25 & 26 & 27 & 28 & 29 & 30 & 31 & 32  \end{bmatrix}  \\
M_{3}  &= \begin{bmatrix} 1 & 2 & 3 \\ 4& 5 & 6 \\ 7 & 8 & 9 \\ 10 & 11 & 12 \\ 13 & 14 & 15 \\ 16 & 17 & 18 \\ 19 & 20 & 21 \\ 22 & 23 & 24 \\ 25 & 26 & 27 \\ 28 & 29 & 30 \\ 31 & 32 & 33  \end{bmatrix} \\
M_{4} &= \begin{bmatrix} 1 & 2 & \hdots & 250 \\ \vdots & \vdots & \ddots & \vdots \\ 124751 & 124752 & \hdots & 250*500 \end{bmatrix}.
\end{align*}


\bigskip

{\bf{Exercise II Function in Python}}\\

We want to compute the function $f(x) = (\sqrt{1+x}-1)/x$ for different values of $x$ in a neighborhood of 0. We first notice that $f(x)$ can be equivalently written as $f(x) = 1/(\sqrt{1+x}+1)$ and also as $f(x)=1/2 - x/8 + x^2/16 - 5x^3/128 + o(x^4)$.

Create three function handles representing the above definitions of $f(x)$ (\textit{hint}: the term $o(x^4)$ can be neglected in the computation) and, for each function handle,
\begin{enumerate}
\item evaluate $f(x)$ at $X = [10^{-10} \, 10^{-12} \, 10^{-14} \, 10^{-16}]$ using a |for| loop;
\item evaluate $f(x)$ at the same points given in a) using Python vector algebra;
\item display the results and comment on the importance of \textit{round-off errors} for this example;
\item make sure you document your functions correctly;
\item if $x=0$ then your first function raises an error explaining why.
\end{enumerate}

\bigskip

{\bf{Exercise III Matrix-vector multiplication in Python}}\\
\begin{enumerate}
\item Consider the multiplication of a matrix $A \in \mathbb{R}^{m \times n}$ with a vector $v \in \mathbb{R}^n$.  Write a Python file containing a script that:
\begin{itemize}
\item creates a matrix of dimension $m \times n$
\item creates a vector of dimenstion $n$
\item define a function that computes $A v$ by using two nested loops
\end{itemize}
\end{enumerate}

\bigskip

{\bf{Exercise IV Matrix-vector multiplication with NumPy}}\\
\begin{enumerate}
\item Consider the same operation as in the previous exercice, the
  multiplication of a matrix $A \in \mathbb{R}^{m \times n}$ with a
  vector $v \in \mathbb{R}^n$.  Compute matrix-vector multiplication by using numpy library:
\begin{itemize}
\item create a matrix of dimension $m \times n$, a vector of dimenstion $n$
\item define a function that computes $A v$ by using two nested loops
\item compare the performance obtained for different values of $m$ and $n$ between the two nested loops code and the code using numpy library and draw a plot displaying the obtained performance
\end{itemize}
\end{enumerate}

{\bf{Exercise V Hello world with Python and MPI}}\\
Execute the following simple code on 4 processors several times. 
\begin{verbatim}
from mpi4py import MPI 
import numpy as np  

comm = MPI.COMM_WORLD
rank = comm.Get_rank()
print("I am rank = ", rank )
\end{verbatim}
To execute this code, do
\begin{verbatim}
$ mpiexec -n 4 python script.py
\end{verbatim}
Observe the order in which the prints take place.
\end{document}


